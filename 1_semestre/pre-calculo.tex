\documentclass[a4paper,12pt]{article}\begin{document}
\Large{Pr\'e C\'alculo 2014.01}\\
\underline{\large{Conjuntos:}}\\
\normalsize{S\~ao exemplos de conjuntos:\\	
a) N = \{0, 1, 2, 3 ...\}\\
b) {janeiro, fevereiro}\\
c) Z = \{...-3, -2, -1, 0, +1, +2, +3...\}\\
d) $Q = \{\frac{a}{b} : a, b \in R\}$\\
e) R = conjunto dos n\'umeros reais.\\
f) C = conjunto dos n\'umeros complexos ${a + bi : a,b \in R}$\\
g) Mn(R) = conjuntos de matrizes com entradas reais.\\
h) \O = conjunto vazio. \{\}\\
\underline{obs:}\\
a) Um elemento $x$ pertence ao conjunto $A$ e escrevemos $x \in A$ se $x$ for elemento de $A$\\
b) Um elemento $x$ n\~ao pertence ao conjunto $A$ e escrevemos $x \notin A$ se $x$ n\~ao for elemento de $A$.\\\\
exemplos:\\
a) $-2 \notin N$\\
b) $-2 \in C$\\\\
\underline{Def:} 
Sejam $A$ e $B$ dois conjuntos. Dizemos que $A$ está contido em $B$, e escrevemos $A \subset B$, se todos os eementos de $A$ também s\~ao elementos de $B$.\\
exemplos:\\
a) $N \subset R$\\       
b) $N \subset Z$\\
c) $N \subset Q$\\
d) \O \'e subconjunto de qualquer conjunto.\\\\
Dem:\\
$A$ \'e um conjunto qualquer.Temos duas opc\~oes: $\O \subset A$ ou $\O \not\subset A$. 
Vamos mostrar que a opc\~ao $\O \not\subset A$ n\~ao acontece. 
Para tanto, vamos supor que $\O \not\subset A$. Ent\~ao existe um elemento $x \in \O$ tal que $x \in A$. 
Por absurdo concluimos que n\~ao \'e verdade que $\O \not\subset A$. Portanto, $\O \subset A$. \\\\
e) $A = \{\{1\}, \{2, 3\}\}$. Os elementos de A s\~ao $\{1\}$ e $\{2, 3\}$. Note que $\{2, 3\} \in A$ mas $\{3\} \notin A$.\\\\
\underline{Def:}
Dizemos que dois conjuntos $A$ e $B$ s\~ao iguais se $A$ e $B$ tiverem os mesmos elementos.\\
Obs: $A=B$ se e somente se $A \subset B$ e $B \subset A$.\\
Dem:\\
$(\Rightarrow)$ Estamos assumindo que $A = B$. Vamos provar que $A \subset B$ e $B \subset A$. Primeiro vamos mostrar que $B \subset A$. Fixe um elemento $x \in B$. Como $A = B$ ent\~ao $x \in A$. Isto significa que $B \subset A$. De forma an\'aloga prova-se que $A \subset B$.\\
$(\Leftarrow)$ Estamos assumindo que $A\subset B$ e $B \subset A$. Vamos demonstrar que $A=B$. Para tanto, vamos supor que não é verdade que $A=B$. Portanto existe elemento $x \in A : x\notin B$ ou $x \in B : x\notin A$. Note que isso não acontece pois $B \subset A$. Da mesma forma a opc\~ao $x \in B : x \notin A$ também não acontece pois $B \subset A$. Ent\~ao a tese est\'a incorreta. Portanto $A  \subset B$.\\\\
\underline{exemplos:}\\
a) $A=\{1, 2, 3\}, B=\{2, 1, 3\}$ Nesse caso $A=B$.\\
b) $A=\{x \in R : x \geq \frac{1}{3}\}, B=\{x \in R : x \geq \frac{1}{2}\}$. Neste caso $A \neq B$.\\
c) $A=\{x \in Z : x \geq \frac{1}{3}\}, B=\{x \in Z : x \geq \frac{1}{2}\}$. Neste caso $A=B$.

\underline{Def:} Seja $A$ um conjunto. O conjunto das partes de $A$, denotado por $P(A)$, é o conjunto de todos os subconjuntos de $A$.\\\\
\underline{exemplos:}\\
a) $A=\{2\}$, $P(A)=\{\O, \{2\}\}$\\
b) $A=\{a. 3\}$, $P(A)=\{\O, \{a\}, \{3\}, \{a, 3\}\}$\\






}\end{document}