\documentclass[10pt,a4paper]{article}
\usepackage[utf8]{inputenc}
\usepackage{amsmath}
\usepackage{amsfonts}
\usepackage{amssymb}
\begin{document}

\begin{flushleft}
\underline{Funções }\\
\end{flushleft}
\underline{Def:} Seja A e B conjuntos n\~ao vazios, dizemos que uma relaç\~ao de A em B \'e uma funç\~ao se:\\
P1) $(\forall x \in A \exists y \in B:(x,y)\in R)$\\
P2) $(\forall x \in A \exists ! y \in B:(x,y)\in R)$\\\\
Exemplos:\\
a)$A=\{1, 2, 3\}, B=\{3, 4\}, R=\{(1,3), (2,4), (1,4)\}$ \\
	Note que R n\~ao \'e uma funç\~ao (P2).\\
b)$A=N, B=N, R=\{(n+1, n):n\in N\} = \{(1,0), (2,1), (3,2)\}$\\
N\~ao \'e uma funç\~ao pois $A y\in N:(0,y) \in N$.\\
c)$A=N*, B=N, R=\{(n+1, n):n \in N\}$ \'e uma funç\~ao.\\
d)$A=\{1,2,3\}, B=\{4,5,6\}, R=\{(1,4),(2,4),(3,5)\}$ \'e uma funç\~ao.\\
\\
Obs: Seja R uma relaç\~ao de A em B ent\~ao R \'e o subconjunto de A em B $(\forall x \in A \exists !y \in B:(x,y) \in R)$.\\
Ent\~ao como $(\forall x \in A \exists !y \in B:(x,y) \in R)$, denotamos y por $f(x)$.\\
Ent\~ao os pares $(x,y)\in R$ podem ser denotados por $f:A\longrightarrow B$.\\
Por exemplo, quando queremos $f:N\longrightarrow N n\longmapsto n^2$ estamos nos referindo à seguinte relaç\~ao: $R=\{(n, n^2):n\in N)\}$.\\
 \\
 Outra notaç\~ao, por exemplo, $f:N\longrightarrow N$ \'e $n\longmapsto n^2$. $f:N\longrightarrow N, f(n) = n^2$.\\
 \\
 \underline{Def:} Seja $f:A\longrightarrow B$ uma funç\~ao:\\
 1) O domínio de f \'e A. A notaç\~ao \'e $Dom(f)=A$\\
 2) O contra-domínio \'e B.\\
 3) A imagem de f \'e o conjunto $\{f(x):x\in A\}$. A notaç\~ao para a imagem de f \'e $Im(f)$.\\
\\
\underline{Def:}Seja $f:A\longrightarrow B$ uma funç\~ao e seja $X \ \subseteq A$. A imagem de X por f \'e o conjunto $\{f(x):x\in X\}$ denotado por $f(X)$.\\
Por exemplo:\\
$f:N\longrightarrow N n\longmapsto n^2$, $X=\{1,2,3\}$.\\
No caso, $f(X) = \{1,4,9\}$.
\end{document}
